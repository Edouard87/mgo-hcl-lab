\documentclass{chemlab}
\usepackage{hyperref}
\usepackage{fancyhdr}
\usepackage{graphicx}
\usepackage{amssymb}
\usepackage{longtable}
\usepackage{booktabs}
\usepackage{gensymb}

\fancyhead[L,C, R]{}
\fancyhead[C]{Edouard Des Parois Perrault}
\fancyfoot[C]{\includegraphics[scale=0.15]{template_images/edouard-seal.png}}
\fancyfoot[R]{\thepage}
\renewcommand{\headrulewidth}{0.4pt}
\renewcommand{\footrulewidth}{2pt}
\renewcommand{\headheight}{15pt}
\renewcommand{\footskip}{80pt}

\pagestyle{fancy}

\title{Heat of Reaction Lab Report}
\subject{Chemistry}
\teacher{Marguerite Comley}
\author{Edouard Des Parois Perrault}
\date{18/02/2021}
\begin{document}
    \maketitle
        \tableofcontents
        \hypertarget{is-the-reaction-endothermic-or-exothermic}{%
\section{Is the Reaction Endothermic or
Exothermic?}\label{is-the-reaction-endothermic-or-exothermic}}

The reaction is exothermic. \(\Delta H\) is therefore negative.

\hypertarget{data-table}{%
\section{Data Table}\label{data-table}}

\begin{longtable}[]{@{}lc@{}}
\toprule
\textbf{Data} & \textbf{Value}\tabularnewline
\midrule
\endhead
Mass \ch{MgO} & 2.8054 \si{g}\tabularnewline
Initial Temperature (\textdegree C) & 25.01\tabularnewline
Final Temperature (\textdegree C) & 46.33\tabularnewline
\bottomrule
\end{longtable}

\hypertarget{total-heat-released}{%
\section{Total Heat Released}\label{total-heat-released}}

\begin{eqnarray}
  Q_{env} & = & mc\Delta t \\
          & = & 100 \cdot 4.18 \cdot 21.29 \\
          & = & 8899.22~\si{J}
\end{eqnarray}

\hypertarget{moles-of-mgo}{%
\section{\texorpdfstring{Moles of
\(MgO\)}{Moles of MgO}}\label{moles-of-mgo}}

\begin{eqnarray}
  n & = & \frac{m}{mm} \\
    & = & \frac{2.0845}{40.304} \\
    & = & 0.0517~\si{mol}
\end{eqnarray}

\hypertarget{calculating-delta-h_rxm}{%
\section{\texorpdfstring{Calculating
\(\Delta H_{rxm}\)}{Calculating \textbackslash Delta H\_\{rxm\}}}\label{calculating-delta-h_rxm}}

\begin{eqnarray}
  \Delta H & = & -\frac{Q_{env}}{0.0517} \\
           & = & -\frac{8899.22}{0.0517} \cdot \frac{1}{1000} \\
           & = & 172.13 \si{kJ}
\end{eqnarray}

\hypertarget{table-of-results}{%
\section{Table of Results}\label{table-of-results}}

\begin{longtable}[]{@{}ccccc@{}}
\toprule
\begin{minipage}[b]{0.17\columnwidth}\centering
\textbf{Mass of the Rxn Mixture}\strut
\end{minipage} & \begin{minipage}[b]{0.17\columnwidth}\centering
\textbf{\(\Delta T\)}\strut
\end{minipage} & \begin{minipage}[b]{0.17\columnwidth}\centering
\textbf{Total Heat Released}\strut
\end{minipage} & \begin{minipage}[b]{0.17\columnwidth}\centering
\textbf{\si{mol} \ch{MgO}}\strut
\end{minipage} & \begin{minipage}[b]{0.17\columnwidth}\centering
\textbf{\(\frac{\Delta H_{rxn}}{\si{mol}}\)}\strut
\end{minipage}\tabularnewline
\midrule
\endhead
\begin{minipage}[t]{0.17\columnwidth}\centering
100 \si{g}\strut
\end{minipage} & \begin{minipage}[t]{0.17\columnwidth}\centering
21.3 \(\degree C\)\strut
\end{minipage} & \begin{minipage}[t]{0.17\columnwidth}\centering
\(8.90 \cdot 10^3 \si{J}\)\strut
\end{minipage} & \begin{minipage}[t]{0.17\columnwidth}\centering
0.0517 \si{mol}\strut
\end{minipage} & \begin{minipage}[t]{0.17\columnwidth}\centering
-172 \si{kJ}\strut
\end{minipage}\tabularnewline
\bottomrule
\end{longtable}
\end{document}